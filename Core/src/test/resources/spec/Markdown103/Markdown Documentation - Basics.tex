\chapter{Markdown: Basics}

\section{Getting the Gist of Markdown's Formatting Syntax}
------------------------------------------------

This page offers a brief overview of what it's like to use Markdown. The \gbturn{syntax page} provides complete, detailed documentation for every feature, but Markdown should be very easy to pick up simply by looking at a few examples of it in action. The examples on this page are written in a before/after style, showing example syntax and the HTML output produced by Markdown.

It's also helpful to simply try Markdown out; the \gbturn{Dingus} is a web application that allows you type your own Markdown-formatted text and translate it to XHTML.

\textbf{Note:} This document is itself written using Markdown; you can \gbturn{see the source for it by adding ‘.text’ to the URL}.

\section{Paragraphs, Headers, Blockquotes}

A paragraph is simply one or more consecutive lines of text, separated by one or more blank lines. (A blank line is any line that looks like a blank line -- a line containing nothing spaces or tabs is considered blank.) Normal paragraphs should not be intended with spaces or tabs.

Markdown offers two styles of headers: \emph{Setext} and \emph{atx}. Setext-style headers for \lstinline{<h1\textgreater} and \lstinline{<h2\textgreater} are created by ‘‘underlining’’ with equal signs (\lstinline{=}) and hyphens (\lstinline{-}), respectively. To create an atx-style header, you put 1-6 hash marks (\lstinline{\#}) at the beginning of the line -- the number of hashes equals the resulting HTML header level.

Blockquotes are indicated using email-style ‘\lstinline{\textgreater}’ angle brackets.

Markdown:
\begin{verbatim}
  A First Level Header
====================

A Second Level Header
---------------------

Now is the time for all good men to come to
the aid of their country. This is just a
regular paragraph.

The quick brown fox jumped over the lazy
dog's back.

\#\#\# Header 3

\textgreater This is a blockquote.
\textgreater 
\textgreater This is the second paragraph in the blockquote.
\textgreater
\textgreater \#\# This is an H2 in a blockquote
\end{verbatim}

Output:
\begin{verbatim}
  <h1\textgreaterA First Level Header</h1\textgreater

<h2\textgreaterA Second Level Header</h2\textgreater

<p\textgreaterNow is the time for all good men to come to
the aid of their country. This is just a
regular paragraph.</p\textgreater

<p\textgreaterThe quick brown fox jumped over the lazy
dog's back.</p\textgreater

<h3\textgreaterHeader 3</h3\textgreater

<blockquote\textgreater
    <p\textgreaterThis is a blockquote.</p\textgreater
    
    <p\textgreaterThis is the second paragraph in the blockquote.</p\textgreater
    
    <h2\textgreaterThis is an H2 in a blockquote</h2\textgreater
</blockquote\textgreater
\end{verbatim}


### Phrase Emphasis ###

Markdown uses asterisks and underscores to indicate spans of emphasis.

Markdown:

    Some of these words *are emphasized*.
    Some of these words _are emphasized also_.
    
    Use two asterisks for **strong emphasis**.
    Or, if you prefer, __use two underscores instead__.

Output:

    <p>Some of these words <em>are emphasized</em>.
    Some of these words <em>are emphasized also</em>.</p>
    
    <p>Use two asterisks for <strong>strong emphasis</strong>.
    Or, if you prefer, <strong>use two underscores instead</strong>.</p>
   


## Lists ##

Unordered (bulleted) lists use asterisks, pluses, and hyphens (`*`,
`+`, and `-`) as list markers. These three markers are
interchangable; this:

    *   Candy.
    *   Gum.
    *   Booze.

this:

    +   Candy.
    +   Gum.
    +   Booze.

and this:

    -   Candy.
    -   Gum.
    -   Booze.

all produce the same output:

    <ul>
    <li>Candy.</li>
    <li>Gum.</li>
    <li>Booze.</li>
    </ul>

Ordered (numbered) lists use regular numbers, followed by periods, as
list markers:

    1.  Red
    2.  Green
    3.  Blue

Output:

    <ol>
    <li>Red</li>
    <li>Green</li>
    <li>Blue</li>
    </ol>

If you put blank lines between items, you'll get `<p>` tags for the
list item text. You can create multi-paragraph list items by indenting
the paragraphs by 4 spaces or 1 tab:

    *   A list item.
    
        With multiple paragraphs.

    *   Another item in the list.

Output:

    <ul>
    <li><p>A list item.</p>
    <p>With multiple paragraphs.</p></li>
    <li><p>Another item in the list.</p></li>
    </ul>
    


### Links ###

Markdown supports two styles for creating links: *inline* and
*reference*. With both styles, you use square brackets to delimit the
text you want to turn into a link.

Inline-style links use parentheses immediately after the link text.
For example:

    This is an [example link](http://example.com/).

Output:

    <p>This is an <a href="http://example.com/">
    example link</a>.</p>

Optionally, you may include a title attribute in the parentheses:

    This is an [example link](http://example.com/ "With a Title").

Output:

    <p>This is an <a href="http://example.com/" title="With a Title">
    example link</a>.</p>

Reference-style links allow you to refer to your links by names, which
you define elsewhere in your document:

    I get 10 times more traffic from [Google][1] than from
    [Yahoo][2] or [MSN][3].

    [1]: http://google.com/        "Google"
    [2]: http://search.yahoo.com/  "Yahoo Search"
    [3]: http://search.msn.com/    "MSN Search"

Output:

    <p>I get 10 times more traffic from <a href="http://google.com/"
    title="Google">Google</a> than from <a href="http://search.yahoo.com/"
    title="Yahoo Search">Yahoo</a> or <a href="http://search.msn.com/"
    title="MSN Search">MSN</a>.</p>

The title attribute is optional. Link names may contain letters,
numbers and spaces, but are *not* case sensitive:

    I start my morning with a cup of coffee and
    [The New York Times][NY Times].

    [ny times]: http://www.nytimes.com/

Output:

    <p>I start my morning with a cup of coffee and
    <a href="http://www.nytimes.com/">The New York Times</a>.</p>


### Images ###

Image syntax is very much like link syntax.

Inline (titles are optional):

    ![alt text](/path/to/img.jpg "Title")

Reference-style:

    ![alt text][id]

    [id]: /path/to/img.jpg "Title"

Both of the above examples produce the same output:

    <img src="/path/to/img.jpg" alt="alt text" title="Title" />



### Code ###

In a regular paragraph, you can create code span by wrapping text in
backtick quotes. Any ampersands (`&`) and angle brackets (`<` or
`>`) will automatically be translated into HTML entities. This makes
it easy to use Markdown to write about HTML example code:

    I strongly recommend against using any `<blink>` tags.

    I wish SmartyPants used named entities like `&mdash;`
    instead of decimal-encoded entites like `&#8212;`.

Output:

    <p>I strongly recommend against using any
    <code>&lt;blink&gt;</code> tags.</p>
    
    <p>I wish SmartyPants used named entities like
    <code>&amp;mdash;</code> instead of decimal-encoded
    entites like <code>&amp;#8212;</code>.</p>


To specify an entire block of pre-formatted code, indent every line of
the block by 4 spaces or 1 tab. Just like with code spans, `&`, `<`,
and `>` characters will be escaped automatically.

Markdown:

    If you want your page to validate under XHTML 1.0 Strict,
    you've got to put paragraph tags in your blockquotes:

        <blockquote>
            <p>For example.</p>
        </blockquote>

Output:

    <p>If you want your page to validate under XHTML 1.0 Strict,
    you've got to put paragraph tags in your blockquotes:</p>
    
    <pre><code>&lt;blockquote&gt;
        &lt;p&gt;For example.&lt;/p&gt;
    &lt;/blockquote&gt;
    </code></pre>
